Figure X depicts three different genomes from TREATMENT.  The first genome (A, left) is the ancestral genome used to seed each multicell at the beginning of a replicate.  This genome comprises three different functional blocks separated by a series of nop_x instructions (nop_x has no effect other than to consume a single virtual CPU cycle; it is a placeholder that provides evolution with a "blank tape").  The first block of instructions, shown highlighted in green, allocates additional memory into which instructions for the cell's offspring may be copied.  The second block, highlighted in yellow, performs the NOT logic function and donates any acquired resources to the multicell.  Finally, the third block, again highlighted in green, copies the parent cell's genome into the memory allocated for the cell's offspring.

The second and third genomes in Figure X (B and C, middle and right) are the final genomes along unrelated lines of descent from replicates 3 and 9, respectively.  Genome B, which is 5459 generations removed from the ancestor, uses a strategy that is based solely upon its location in the multicell to fully differentiate into either germ or soma.  Genome C, at generation 4663, uses a strategy that relies upon location and messaging to differentiate.

While the specific strategies and instructions used by genomes B and C differ, they are logically similar.  For example, both genomes contain blocks of instructions that perform cell offspring allocation and cell replication.  Moreover, both genomes contain two blocks of instructions that are critical for germ-soma differentiation.  The first, highlighted in blue-red, is the block propagation conditional.  These instructions test for a genome-specific condition, and if true, block propagation of the cell (i.e., the cell marks itself as a soma).  For example, genome B blocks propagation when the y-component of its x-y location in the multicell is non-zero, while genome C blocks propagation when it receives a message whose data is numerically less than an internally calculated value.  Each genome also contains a block of instructions we call the "soma loop," which is a block of instructions that performs logical functions (shown highlighted in red).  While germ cells may safely execute the instructions in the soma loop, cells that have blocked their propagation execute these instructions in an infinite loop.  That is, once a cell has elected to become a soma, it can never leave the soma loop.

The combination of these two strategies, a condition for blocking propagation and an infinite soma loop, has significant ramifications for the multicell.  Specifically, cells that are locked in their soma loop are no longer able to replicate.  Thus, while soma may be accumulating mutations as a result of their workload, these mutants are not, in fact, competing for space with germ cells--they are simply contributing resources to the multicell.  Moreover, the condition under which a cell chooses to block its propagation implicitly requires a multicell with a large cell population.  E.g., genome B, a location-based strategy, will not block propagation until cells have replicated beyond the first row of the multicell.  This condition is stronger in the case of genome C, which requires neighboring cells to communicate with each other prior to propagation being blocked.